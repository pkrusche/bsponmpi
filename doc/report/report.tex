\documentclass[a4paper]{article}
\usepackage{a4wide}
\usepackage[english]{babel}
\usepackage{subfigure}
\usepackage{graphicx}

\newcommand{\comparisonfigure}[2]{
\begin{figure}
\subfigure{\includegraphics[width=7.5cm]{#1/g#22.eps}} 
\subfigure{\includegraphics[width=7.5cm]{#1/g#24.eps}}\\ 
\subfigure{\includegraphics[width=7.5cm]{#1/g#216.eps}}
\caption{Comparison of BSPonMPI and Oxford BSP Toolset using \emph{#2}
operations on 2, 4 and 16 processors of the computer #1 (from left to right
and top to bottom).} 
\label{fig:#1#2}
\end{figure}
}

\title{Implementing BSPonMPI 0.1 \\ Lessons Learned \& Results}
\author{Wijnand Suijlen}

\begin{document}
\maketitle{}

\abstract{BSPonMPI implements the BSPlib standard and runs on all machines
which have MPI. During the development a few lessons were learned about
writing fast C programs. Additionaly the library is benchmarked and results
show that it can compete with Oxford BSP Toolset. Remarkable is the
observation that the MPI implementation of the Oxford BSP Toolset scales better
then the native implementation. }

\section{Introduction}
BSPonMPI is a platform independent software library for developing parallel
programs. It implements the BSPlib standard \cite{bsplib}
and runs on all machines which have MPI \cite{mpi}. This last property is the
main feature of this library and with this feature it distinguishes itself
from the two major BSP libraries: Oxford BSP Toolset \cite{oxtool} and PUB
\cite{pub}.  Both are implemented for specific hardware platforms (e.g. Cray
T3E or SGI Origin) and they have a platform independent version on top
of MPI.  However the architecture of their software library is optimised for
the use of hardware specific features. Building on top of MPI was never their
primary objective. The mission of BSPonMPI is to be the fastest BSP library
which uses MPI.

During the implementation of this library and the quest for better performance
I learned a few things. After finishing the implementation they seem obvious,
but they were not that obvious beforehand.  This report is about the
experience gained while writing this library and an evaluation of the
performance.  Code documentation, design issues and library usage can be found
on the project homepage
\cite{homepage}. 

Throughout this document I use some terminology which is common in the field
of BSP programming. Here is short glossary
\begin{description}
\item[$g$] Throughput parameter. It stands for the costs of sending one extra
data element (e.g. measured in seconds per byte ).
\item[$l$] Latency parameter. It denotes the minimum costs of communication
or, more specific, the cost of one \verb|bsp_sync()| in which $0$ bytes are
communicated (e.g. measured in seconds).
\item[$h$-relation] Communication pattern. $h$ stands for the amount
of data sent by each processor. If this differs among processors, it denotes
the maximum. An $h$-relation is full, if all processors send at least one data
element. 
\end{description}

\section{Lessons learned}
Before this project my only experience in High Performance Computing (HPC) was
a computer exercise about computing the homogeneous Poisson equation on a
rectangular grid using Fortran. In this exercise the data structure was
obvious and very simple. Additionally Fortran does not allow many complex data
structures, because it is not aware of pointers. In C anything is possible and
therefore it may be tempting to use the full range of C's pointer abilities.

In fact: I was tempted. Especially because I am used to an object oriented
programming style. If I needed to store a set of variables of which I thought
they belonged to each other, I defined a \verb|struct| for it and gave it an
address: statically or dynamically.  So the first time implementing the
communication buffer I constructed it as a linked list, where memory is
allocated dynamically for each communication instruction object. This solution
proved to be very slow for three different reasons:
\begin{enumerate}
\item In order to send the communication buffer using \verb|MPI_Alltoallv|, it
still had to be copied to an array.
\item The communication data could be scattered throughout the
heap\footnote{In general a program has two memory regions in which it stores
its data: the stack and the heap. The stack contains function parameters,
return values, variables and arrays of fixed size. The heap contains all
dynamically allocated data.} may result in a lot of cache misses. 
\item calls to malloc and free are not very cheap.
\end{enumerate}

The greatest increase of performance was gained by doing the memory management
myself: allocate an array at the start of the program and resize it if
necessary. On Teras using 2 processors (see \ref{subsec:testenv}) the BSP
throughput parameter $g$ was reduced by 50\% (see figure \ref{fig:step0-2}) as
memory expansions now only happen a few times during the program instead of
at every addition of a communication instruction. 

\begin{figure}[htbp]
\includegraphics[width=10cm]{teras/step0-2.eps}
\caption{Compares Oxford BSP Toolset to three versions of BSPonMPI:
\textbf{iteration 0} is an implementation having a linked list as communication
buffer;
\textbf{iteration 1} is the first implementation having arrays as communication
buffer;
\textbf{iteration 2} is the current version in which some other optimisations
have been done
}
\label{fig:step0-2}
\end{figure}

Two lessons are learned here:
\begin{enumerate}
\item \emph{Keep it simple}: Implement your algorithms as you would do in a
basic programming language like Fortran or Basic. In this case arrays and
array operations are used instead of linked list  and complex pointer
operations.
\item \emph{Put all complex / slow code in exceptions which happen seldom}: 
In this case only a few times memory was expanded instead of at each action.
\end{enumerate}
This does not only result in faster code, it also provides the compiler with
more opportunity for optimisation and makes the code more readable.
Applying these lessons on other parts of the code reduced $g$ again by 50 \%
(see figure \ref{fig:step0-2}). 

If you follow these rules, you almost automatically obey the standard
optimisations rules:
\begin{itemize}
\item When optimising code, focus on the lines which are executed most
frequently, e.g.: optimise the body of a loop. 
\item Limit the use of branch statements, e.g. \verb|if| or \verb|switch|
\item Limit calls to \verb|extern| functions 
\item Access data using aligned addresses.
\item Use simple language, i.e. use 
\begin{verbatim}
for (i = 0; i < 10; i ++)
  sum += a[i];
\end{verbatim}    
in stead of
\begin{verbatim}
b = a + 9;
while (b >= a) 
  sum += *b;
\end{verbatim}  
\item Supply the compiler with hints, i.e. use \verb|const|, \verb|inline| and
\verb|restrict|.
\end{itemize}   

\section{Results}
\subsection{Test environment}
\label{subsec:testenv}
The performance of BSPonMPI 0.1 has been measured on two different machines:
\begin{itemize}
\item Teras, which is a SGI Origin 3800. On this machine a version of Oxford
BSP Toolset v1.4 is installed which is specially built for this hardware
platform.
\item Aster, which is a SGI Altix 3700. This machine has a version
of Oxford BSP Toolset v1.4 built on top of MPI.
\end{itemize}
The benchmark program used is a modified version of `bench' which is part of
'BSPedupack' \cite{bspedupack}. The modified version can be found at the
BSPonMPI homepage \cite{homepage}. Whereas the original version only
measured the speed of a \emph{put} operations, the modified version can also
measure using \emph{get} and \emph{send} operations. Output of `bench' is
provided in figures \ref{fig:terasput}--\ref{fig:astersend} on the following
pages.

\comparisonfigure{teras}{put}
\comparisonfigure{teras}{get}
\comparisonfigure{teras}{send}

\comparisonfigure{aster}{put}
\comparisonfigure{aster}{get}
\comparisonfigure{aster}{send}
\newpage

The purpose of BSPonMPI is to beat Oxford BSP Toolset on Aster as it was
supposed to be suboptimal. BSPonMPI does not not succeed in that when
considering \emph{put}s only, but astonishingly it beats Oxford on both
machines when measuring using \emph{get}s and \emph{send}s. BSPonMPI was
designed to process \emph{put}s, \emph{get}s and \emph{send}s in an equivalent
manner. The benchmark results confirm this as experiments on equally sized
$h$-relations show equal timings. Apparently Oxford BSP Toolset 
processes them quite differently.

You may also notice that performance differences becomes less apparent when
the number of processors increases. From
figures~\ref{fig:terasput}--\ref{fig:astersend} it can be seen that latency
increases very fast, while the throughput parameter $g$ shows a more moderate
increase. Obviously the latency becomes more and more dominant when the
number of processors is scaled up. In figure~\ref{fig:latency} the latency is
shown for a varying number of processors. It seems that BSPonMPI and its
competitor perform equally on Teras, whereas BSPonMPI performs always worse on
Aster. Again something very strange happened here. 

\begin{figure}
\subfigure{\includegraphics[width=7cm]{teras/latency.eps}}
\subfigure{\includegraphics[width=7cm]{aster/latency.eps}}
\caption{This figure shows the cost of a full $h$-relation while scaling up
the number of processors. On the left results on Teras are shown, on the right
Aster}
\label{fig:latency}
\end{figure}

\section{Conclusion}
The pros and cons of BSPonMPI are
\begin{description}
\item[-] \emph{put} operations are not yet very quick
\item[-] high latency on non-full $h$-relations. 
\item[-] it is not yet capable of detecting BSP programming errors. Up to now
all effort has been directed in implementing a fast library; not a robust one
\item[+] \emph{get} and \emph{send} operations are very fast.
\item[+] The source code is platform independent.  
\item[+] It transforms any BSP program into an MPI program. All MPI tools,
such as profilers, will work with them.
\end{description}

Although this implementation is not yet fully optimised, we see that BSPonMPI
can already compete with the native and MPI versions of Oxford BSP Toolset.
Which library is best, depends on the application. Remarkable is the fact that
BSPonMPI performs best when compared to the native version of Oxford BSP
Toolset. As BSPonMPI is the constant factor in this comparison, one may wonder
whether the MPI version of Oxford BSP Toolset performs better than the native
version.

I expect to gain more performance in a subsequent version of BSPonMPI,
because: 
\begin{description}
\item[latency]  $l$ can be reduced by 25\%  merging two
\verb|MPI_Alltoall()|'s which communicate buffer sizes. More may be
gained by determining whether it is really necessary to communicate buffer
sizes.  
\item[throughput] $g$ can be reduced by 30\% by changing the communication
buffer. 
\end{description}
This may yield a library which scales much better and beats the native and MPI
versions of Oxford BSP Toolset.

\begin{thebibliography}{1}
\bibitem[1]{pub} Paderborn University BSP-library\\
             \verb|http://wwwcs.uni-paderborn.de/~bsp|
\bibitem[2]{mpi} MPI 1.1 Standard \\
            \verb|http://www.mpi-forum.org/docs/docs.html|
\bibitem[3]{bspedupack} Rob Bisseling. BSPedupack v1.0 \\
        \verb|http://www.math.uu.nl/people/bisselin/software.html|
\bibitem[4]{oxtool} Jonathan Hill. Oxford BSP Toolset v1.4 \\
        \verb|http://www.bsp-worldwide.org/implmnts/oxtool|
\bibitem[5]{bsplib} Jonathan Hill, Bill McColl, Dan Stefanescu, Mark Goudreau,
Kevin Lang, Satish Rao, Torsten Suel, Thanasis, Tsantilas and Rob Bisseling.
BSPlib: The BSP programming library. \textit{Parallel Computing,}
24:1947--1980, 1998.
\bibitem[6]{leslearned} Jonathan M. D. Hill, David B. Skillicorn.
Lessons Learned from Implementing BSP. \textit{HPCN Europe 1997,} 762--771
\bibitem[7]{homepage} Wijnand Suijlen. BSPonMPI homepage\\
        \verb|http://bsponmpi.sourceforge.net|
\end{thebibliography}

\end{document}

